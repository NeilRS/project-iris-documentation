Project Iris is a software solution that will work in conjunction with an Intel RealSense SR300 camera to provide 3D gaze tracking information to a host computer. Ultimately the target for this solution are developers looking to incorporate 3D gaze tracking into their existing applications, or develop new applications around the added functionality Project Iris can provide. Project Iris will be open source and provided via a GitHub repository to any developer who wishes to use it.

\subsection{Purpose and Use}
Project Iris will provide 3D gaze tracking information to a host computer in the form of x and y coordinates similar to the way an operating system might provide mouse coordinates to a program. To accommodate this Project Iris will also provide a calibration utility and a screen overlay to indicate where the user is currently looking. This is to be used in conjunction with other software which is outside the scope of Project Iris, but could include mouse emulation, gaming, and efficiency/process improvements for desktop applications. In addition project Iris will also provide a demo program so that users and developers can get a sense of what the camera is seeing.

\subsection{Intended Audience}
Project Iris is not intended to be a standalone solution, but a tool developers of other software can include to create new and immersive user experiences. Game developers, application developers, and developers of accessibility software are all part of the target audience for Project Iris.

\subsection{Features \& Functions}
The code provided by project Iris will include a calibration utility, an onscreen indicator of where the user is currently looking, and an API that will return the x and y coordinates of the user's current gaze position on the screen. Project Iris will be Open Source software intended for use on a Windows operating system and will require an Intel SR300. Additionally project Iris will provide software that allows users and developers tu run a demo mode program to view the output of the camera's video feeds to get a sense of what the application is doing. 

\subsection{External Inputs \& Outputs}
Project Iris will work by using the data gathered from an Intel RealSense SR300 camera mounted as near the horizontal center and vertical top of a computer monitor as possible. The system will expect to be able to reconcile a user's eyes and facial landmarks from a distance of .2 - 1.5 meters. After a calibration procedure this data will be translated into x and y coordinates for use in other applications. In addition to the calibration utility and coordinate outputs via an API, the system will allow the developer to toggle on an indicator to visualize the current gaze position.

\subsection{Product Interfaces}
Project Iris will provide a calibration utility that will place a dot at different positions on the computer monitor (figure 2) and then use this data to output an estimate of the user's current gaze position relative to the screen. The position will be provided via an API call in the form of a pair of integers representing the estimated x and y coordinates of the user's gaze. In addition, the system will provide a toggle via the API to turn on a translucent dot that will allow for the visualization of the current estimated gaze position (figure 3).